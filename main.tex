\documentclass{article}
\usepackage{graphicx} % Required for inserting images
\usepackage[english]{babel}
\usepackage{amssymb}
\usepackage{amsthm}
\usepackage{enumitem} 
\usepackage{amsmath}
\usepackage{amsfonts}
\usepackage{tikz-cd}
\usetikzlibrary{matrix}
\usepackage{mathtools}
\usepackage[a4paper, total={6in, 8in}]{geometry}
\usepackage{cite}
\graphicspath{ {./images/} }
\setcounter{section}{0}

\newtheorem{theorem}{Theorem}[section]
\newtheorem{definition}[theorem]{Definition}
\newtheorem{lemma}[theorem]{Lemma}
\newtheorem{proposition}[theorem]{Proposition}
\newtheorem{corollary}[theorem]{Corollary}
\newtheorem{example}[theorem]{Example}
\newtheorem{remark}[theorem]{Remark}
\newtheorem{exercise}{Exercise}[subsection]
\title{Algebra from a Categorical Perspective}
\author{Saxon Supple}
\date{February 2026}

\begin{document}

\maketitle

\section{Introduction}
This article will give categorical definitions of the standard objects from algebra, by way of defining universal properties which determine the objects up to unique isomorphism. For a given object $A$ of interest, the general procedure will be to first define an object $B$ as an object which satisfies a specific universal property, then show that $A$ satisfies that universal property, and then show that all pairs of objects $B$ which satisfy the universal property are related by a unique isomorphism. This will then demonstrate that the universal property determines $A$ up to unique isomorphism. We will then apply these universal properties to prove additional useful categorical results, such as the functoriality of the abelianization of a group.
\section{Preliminary Categorical Constructions}
\begin{definition}
Let $\mathbf{C}$ be a category and let $f_1,f_2:X\to Y\in\text{Hom}_\mathbf{C}(X,Y)$. Then a morphism $j:W\to X\in\text{Hom}_\mathbf{C}(W,X)$ is an equaliser of $f_1$ and $f_2$ if $f_1\circ j=f_2\circ j$ and given any morphism $g:T\to X$ with $f_1\circ g=f_2\circ g$, there is a unique morphism $h:T\to W$ such that $j\circ h=g$.\[
\begin{tikzcd}
W \arrow[r, "j"] & X \arrow[r, shift left, "f_1"] \arrow[r, shift right, swap, "f_2"] & Y \\
T \arrow[u, "\exists! h"] \arrow[ur, swap, "\forall g"] & & 
\end{tikzcd}
\]
\end{definition}
\begin{theorem}
Let $\mathbf{C}$ be a category, and suppose that $j:W\to X$ and $j':W'\to X$ are both equalizers of morphisms $f_1,f_2\in\text{Hom}_\mathbf{C}(X,Y)$. Then there exists a unique isomorphism $h:W\to W'$ such that $j=j'\circ h$.
\end{theorem}
\begin{proof}
Since $j'$ is an equalizer and $f_1\circ j=f_2\circ j$, there exists a unique morphism $h:W\to W'$ such that $j=j'\circ h$. Similarly, there exists a unique morphism $h':W'\to W$ such that $j'=j\circ h'$. Hence, we have $j'=j'\circ (h\circ h')$. If we then let $T=W'$ and $g=j'$, since we have that both $j'\circ\text{id}_{W'}=j'$ and $j'\circ(h\circ h')=j'$, it follows from uniqueness that $h\circ h'=\text{id}_{W'}$. Similarly, $h'\circ h=\text{id}_W$. Hence, $h$ is an isomorphism.
\end{proof}
\begin{definition}
Let $\mathbf{C}$ be a category and let $f_1,f_2:X\to Y$ be morphisms. Then a morphism $q:Y\to Z$ is a coequalizer of $f_1$ and $f_2$ if $q\circ f_1=q\circ f_2$ and if $g:Y\to W$ is a morphism with $g\circ f_1=g\circ f_2$, then there is a unique morphism $h:Z\to W$ with $g=h\circ q$.\[\begin{tikzcd}
X \arrow[r, shift left, "f_1"] \arrow[r, shift right, swap, "f_2"] & Y \arrow[r, "q"] \arrow[dr, swap, "\forall g"] & Z \arrow[d, "\exists! h"] \\
& & W
\end{tikzcd}\]
\end{definition}
\begin{theorem}
Let $\mathbf{C}$ be a category, and let $q:Y\to Z$ and $q':Y\to Z'$ both be coequalizers of $f_1,f_2\in\text{Hom}_\mathbf{C}(X,Y)$. Then there is a unique isomorphism $h:Z\to Z'$ such that $q'=h\circ q$.
\end{theorem}
\begin{proof}
Since $q$ is a coequalizer and $q'\circ f_1=q'\circ f_2$, there exists a unique morphism $h:Z\to Z'$ such that $q'=h\circ q$. Similarly, there exists a unique morphism $h':Z'\to Z$ such that $q=h'\circ q'$. Hence, we have $q'=(h\circ h') \circ q'$. If we then let $W=Z'$ and $g=q'$, since we have that both $\text{id}_{Z'}\circ q'=q'$ and $q'=(h\circ h') \circ q'$, it follows from uniqueness that $h\circ h'=\text{id}_{Z'}$. Similarly, $h'\circ h=\text{id}_Z$. Hence, $h$ is an isomorphism.
\end{proof}
\begin{definition}
Let $\mathbf{C}$ be a category, and let $(X_\alpha)_{\alpha\in\mathcal{A}}$ be an indexed collection of objects of $\mathbf{C}$. An object $X$ equipped with morphisms $\text{pr}_\alpha:X\to X_\alpha$ is a product of $(X_\alpha)_{\alpha\in\mathcal{A}}$ if for any object $Z$ and morphisms $f_\alpha:Z\to X_\alpha$, there exists a unique morphism $F:Z\to X$ such that $\text{pr}_\alpha\circ F=f_\alpha\forall\alpha\in\mathcal{A}$.\[
\begin{tikzcd}
& Z \arrow[ldd, "f_1"', bend right] \arrow[rdd, "f_2", bend left] \arrow[d, "\exists! F"] & \\
& X \arrow[ld, "\mathrm{pr}_1"] \arrow[rd, "\mathrm{pr}_2"'] & \\
X_1 & & X_2
\end{tikzcd}
\]
\end{definition}
\begin{theorem}
Let $\mathbf{C}$ be a category, and let $(X_\alpha)_{\alpha\in\mathcal{A}}$ be an indexed collection of objects of $\mathbf{C}$. Suppose $(X,\text{pr}_\alpha)$ and $(X',\text{pr}'_\alpha)$ are both products of $(X_\alpha)_{\alpha\in\mathcal{A}}$. Then there exists a unique isomorphism $h:X\to X'$ such that $\text{pr}_\alpha'\circ h=\text{pr}_\alpha\forall\alpha\in\mathcal{A}$.
\end{theorem}
\begin{proof}
If we let $Z=X'$ and $f_\alpha=\text{pr}_\alpha'$, then there exists a unique morphism $h':X'\to X$ such that $\text{pr}_\alpha'=\text{pr}_\alpha\circ h'$. Similarly, there exists a unique morphism $h:X\to X'$ such that $\text{pr}_\alpha=\text{pr}_\alpha'\circ h$. From this we obtain $\text{pr}_\alpha=\text{pr}_\alpha\circ(h'\circ h)$. Then, if we let $Z=X$ and $f_\alpha=\text{pr}_\alpha$, we see that both $h'\circ h$ and $\text{id}_X$ are candidates for $F$. Hence, by uniqueness, $h'\circ h=\text{id}_X$. Similarly, $h\circ h'=\text{id}_{X'}$. Hence, $h$ is an isomorphism.
\end{proof}
\begin{definition}
Let $\mathbf{C}$ be a category and let $(X_\alpha)_{\alpha\in\mathcal{A}}$ be an indexed collection of objects of $\mathbf{C}$. An object $Y$ along with morphisms $\iota_\alpha:X_\alpha\to Y$ is a coproduct of $(X_\alpha)_{\alpha\in\mathcal{A}}$ if for any object $Z$ with morphisms $f_\alpha:X_\alpha\to Z$, there exists a unique morphism $F:Y\to Z$ such that for all $\alpha\in\mathcal{A}$, we have $F\circ\iota_\alpha=f_\alpha$.\[\begin{tikzcd}
X_1 \arrow[dr, "\iota_1"] \arrow[ddr, bend right, "f_1"'] & & X_2 \arrow[dl, "\iota_2"'] \arrow[ddl, bend left, "f_2"] \\
& Y \arrow[d, "\exists! F"] & \\
& Z & 
\end{tikzcd}\]
\end{definition}
\begin{theorem}
Let $\mathbf{C}$ be a category and let $(X_\alpha)_{\alpha\in\mathcal{A}}$ be objects of $\mathbf{C}$. Suppose that both $(Y,(\iota_\alpha)_{\alpha\in\mathcal{A}})$ and $(Y',(\iota'_\alpha)_{\alpha\in\mathcal{A}})$ are coproducts of $(X_\alpha)_{\alpha\in\mathcal{A}}$. Then there exists a unique isomorphism $h:Y\to Y'$ such that $h\circ\iota_\alpha=\iota'_\alpha\forall\alpha\in\mathcal{A}$.
\end{theorem}
\begin{proof}
If we let $Z=Y'$ and $f_\alpha=\iota_\alpha'$, there exists a unique morphism $h:Y\to Y'$ such that $\iota_\alpha'=h\circ\iota_\alpha\forall\alpha\in\mathcal{A}$. Similarly, there exists a unique morphism $h':Y'\to Y$ such that $\iota_\alpha=h'\circ\iota_\alpha'\forall\alpha\in\mathcal{A}$. Hence, we have $\iota_\alpha=(h'\circ h)\circ\iota_\alpha\forall\alpha\in\mathcal{A}$. If we then let $Z=Y$ and $f_\alpha=\iota_\alpha$, we see that both $h'\circ h$ and $\text{id}_Y$ are candidates for $F$. Hence, by uniqueness, $h'\circ h=\text{id}_Y$. Similarly, $h\circ h'=\text{id}_{Y'}$. Hence, $h$ is an isomorphism.
\end{proof}
\section{Sets}
\subsection{Quotients}
\begin{definition}
Let $X$ be a set, and let $\sim$ be an equivalence relation on $X$. We say that a function $f:X\to Y$ is $\sim$-invariant if $x_0\sim x_1\implies f(x_0)=f(x_1)\forall x_0,x_1\in X$.
\end{definition}
\begin{definition}[Universal property of the quotient set]
Let $X$ be a set and let $\sim$ be an equivalence relation on $X$. A pair $(X',\pi:X\to X')$ is a quotient of $X$  by $\sim$ if $\pi$ is $\sim$-invariant, and given any $\sim$-invariant function $f:X\to Y$, there exists a unique $\overline{f}:X'\to Y$ such that $f=\overline{f}\circ \pi$.
\end{definition}
\begin{theorem}
The canonical projection $\pi:X\to X/\sim:x\mapsto[x]$ satisfies the universal property of the quotient set.
\end{theorem}
\begin{proof}
First note that $\pi$ is clearly $\sim$-invariant. We define $\overline{f}:X/\sim\to Y$ by $\overline{f}(\pi(x))=f(x)$. This is well-defined, since if $x_0\sim x_1$, then $\overline{f}(\pi(x_0))=f(x_0)=f(x_1)=\overline{f}(\pi(x_1))$, and hence $\overline{f}$ is independent of representatives. Uniqueness is also evident from the definition of $\pi$.
\end{proof}
\begin{theorem}
Let $X$ be a set with an equivalence relation $\sim$. Let $(W,\pi_W)$ and $(Z,\pi_Z)$ be two pairs of sets, along with maps $\pi_W:X\to W$ and $\pi_Z:X\to Z$ such that both pairs satisfy the universal property of the quotient set. Then there exists a unique bijection $h:W\to Z$ such that $\pi_Z=h\circ\pi_W$.
\end{theorem}
\begin{proof}
Let $(X',\pi:X\to X')$ be a quotient of $X$ by $\sim$. Let $R\subseteq X\times X$ be defined as $R=\{(x_0,x_1):x_0\sim x_1\}$. Then define $\text{pr}_1:R\to X:(x_0,x_1)\mapsto x_0$ and $\text{pr}_2:R\to X:(x_0,x_1)\mapsto x_1$. We then note that $\sim$-invariance of a function $g:X\to Y$ is equivalent to the property that $g\circ\text{pr}_1=g\circ\text{pr}_2$. We then have that $\pi\circ\text{pr}_1=\pi\circ\text{pr}_2$, and given any $f:X\to Y$ such that $f\circ\text{pr}_1=f\circ\text{pr}_2$, there exists a unique $\overline{f}:X'\to Y$ such that $f=\overline{f}\circ\pi$.\[\begin{tikzcd}
R \arrow[r, shift left, "\text{pr}_1"] \arrow[r, shift right, swap, "\text{pr}_2"] & X \arrow[r, "\pi"] \arrow[dr, swap, "\forall f"] & X' \arrow[d, "\exists! \overline{f}"] \\
& & Y
\end{tikzcd}\] This is identical to saying that $\pi$ is a coequalizer of $\text{pr}_1$ and $\text{pr}_2$ in the category of sets, and hence, $\pi$ is unique up to unique bijection. In particular, there exists a unique bijection $h:W\to Z$ such that $\pi_Z=h\circ\pi_W$.
\end{proof}
\subsection{Products}
\begin{definition}[Universal property of the product of sets]
Let $(S_\alpha)_{\alpha\in\mathcal{A}}$ be an indexed collection of sets. A set $S$ equipped with maps $\text{pr}_\alpha:S\to S_\alpha$ is a product of $(S_\alpha)_{\alpha\in\mathcal{A}}$ if for any set $K$ and maps $\phi_\alpha:K\to S_\alpha$, there exists a unique map $\Psi:K\to S$ such that $\text{pr}_\alpha\circ\Psi=\phi_\alpha\forall\alpha\in\mathcal{A}$.
\end{definition}
\begin{theorem}
Let $(S_\alpha)_{\alpha\in\mathcal{A}}$ be an indexed collection of sets. The set \[\prod_{\alpha\in\mathcal{A}}S_\alpha:=\{x:\mathcal{A}\to\bigcup_{\alpha\in\mathcal{A}}S_\alpha\mid\forall\alpha\in\mathcal{A}:x(\alpha)\in S_\alpha\}\] equipped with maps $\text{pr}_\beta:\prod_{\alpha\in\mathcal{A}}S_\alpha\to S_\beta:x\mapsto x(\beta)$ is a product of $(S_\alpha)_{\alpha\in\mathcal{A}}$.
\end{theorem}
\begin{proof}
Let $K$ be another set and let $f_\beta:K\to S_\beta$ be a collection of maps. There is then a unique map $F:K\to\prod_{\alpha\in\mathcal{A}}S_\alpha$ with $\text{pr}_\beta\circ F=f_\beta$, namely the one given by $F(k)(\alpha)=f_\alpha(k)\forall k\in K,\alpha\in\mathcal{A}$.
\end{proof}
\begin{remark}
Note that if $\mathcal{A}$ is infinite and each $S_\alpha$ is non-empty, then $\prod_{\alpha\in\mathcal{A}}S_\alpha$ is only non-empty if we assume the axiom of choice, since the definition requires us to pick an element from infinitely many non-empty sets.
\end{remark}
\begin{theorem}
Let $(S_\alpha)_{\alpha\in\mathcal{A}}$ be an indexed collection of sets and let $(S,(\text{pr}_\alpha)_{\alpha\in\mathcal{A}})$ and $(S',(\text{pr}'_\alpha)_{\alpha\in\mathcal{A}})$ be two products of $(S_\alpha)_{\alpha\in\mathcal{A}}$. Then there exists a unique bijection $h:S\to S'$ such that $\text{pr}'_\alpha\circ h=\text{pr}_\alpha\forall\alpha\in\mathcal{A}$.
\end{theorem}
\begin{proof}
The product of sets is the product in the category of sets, and so uniqueness follows.
\end{proof}
\subsection{Disjoint Unions}
\begin{definition}[Universal property of disjoint unions]
Let $(S_\alpha)_{\alpha\in\mathcal{A}}$ be an indexed collection of sets. A set $S$ with maps $\iota_\alpha:S_\alpha\to S$ is said to be the disjoint union of $(S_\alpha)_{\alpha\in\mathcal{A}}$, written $S=\coprod_{\alpha\in\mathcal{A}}S_\alpha$ if for any set $K$ with maps $f_\alpha:S_\alpha\to K$, there exists a unique map $F:S\to K$ such that $F\circ\iota_\alpha=f_\alpha\forall\alpha\in\mathcal{A}$.
\end{definition}
\begin{theorem}
Any collection $(S_\alpha)_{\alpha\in\mathcal{A}}$ of sets has a disjoint union $\coprod_{\alpha\in\mathcal{A}}S_\alpha$.
\end{theorem}
\begin{proof}
We can set $\coprod_{\alpha\in\mathcal{A}}S_\alpha=\{(x,\alpha)\in S_\alpha\times\mathcal{A}\}$, and we can define inclusion maps $\iota_\alpha:S_\alpha\to \coprod_{\alpha\in\mathcal{A}}S_\alpha:x\mapsto(x,\alpha)$. Then, if we have another set $K$ and maps $f_\alpha:S_\alpha\to K$, there exists a unique map $F:\coprod_{\alpha\in\mathcal{A}}S_\alpha\to K$ such that $F\circ\iota_\alpha=f_\alpha\forall\alpha\in\mathcal{A}$, namely by setting $F((x,\alpha))=f_\alpha(x)$.
\end{proof}
\begin{theorem}
Let $(S_\alpha)_{\alpha\in\mathcal{A}}$ be an indexed collection of sets, and let $(S,(\iota_\alpha)_{\alpha\in\mathcal{A}})$ and $(S',(\iota'_\alpha)_{\alpha\in\mathcal{A}})$ be two disjoint unions of $(S_\alpha)_{\alpha\in\mathcal{A}}$. Then there exists a unique bijection $h:S\to S'$ such that $h\circ\iota_\alpha=\iota_\alpha'\forall\alpha\in\mathcal{A}$.
\end{theorem}
\begin{proof}
The disjoint union is the coproduct in the category of sets, and so uniqueness up to unique bijection follows.
\end{proof}
\section{Groups}
\subsection{Quotient Groups}
\begin{definition}[Universal property of the quotient group]
Let $G$ be a group, and let $N\trianglelefteq G$ be a normal subgroup. A pair $(Q,q:G\to Q)$, consisting of a group $Q$ and a group homomorphism $q$, is a quotient of $G$ if $N\subseteq\ker q$ and given any group homomorphism $\phi:G\to H$ with $N\subseteq\ker \phi$, there exists a unique group homomorphism $\overline{\phi}:Q\to H$ such that $\phi=\overline{\phi}\circ q$.
\end{definition}
\begin{theorem} Let $G$ be a group and let $N\trianglelefteq G$ be a normal subgroup. Then the canonical quotient homomorphism $\pi:G\to G/N:g\mapsto gN$ is a quotient homomorphism.
\end{theorem}
\begin{proof}
Let $\sim$ be an equivalence relation on $G$ which identifies the fibres of $\pi$; that is, $x\sim y\iff xN=yN$. If $\pi(x)=\pi(y)$, then $x$ and $y$ are in the same coset, so there exists some $n\in N$ such that $x=yn$. Hence, since $N\subseteq\ker\phi$, we have $\phi(x)=\phi(yn)=\phi(y)\phi(n)=\phi(y)$. Then by the universal property of the quotient set, there exists a unique map $\overline{\phi}:G/\sim\to H$ such that $\phi=\overline{\phi}\circ\pi$. In particular, $\overline{\phi}$ is defined as $\overline{\phi}([x])=\overline{\phi}(xN)=\phi(x)\forall x$, and hence $\overline{\phi}$ is simply a map on the cosets of $N$. Furthermore, given $x,y\in G$, we have $\overline{\phi}(xyN)=\phi(xy)=\phi(x)\phi(y)=\overline{\phi}(xN)\overline{\phi}(yN)$, and hence $\overline{\phi}$ is a group homomorphism.
\end{proof}

\begin{corollary}
Let $\phi:G\to H$ be a group homomorphism, and let $N\trianglelefteq G$ be a normal subgroup such that $N\subseteq\ker\phi$. Then $N=\ker\phi$ if and only if $\overline{\phi}:G/N\to H$ is injective.
\end{corollary}
\begin{proof}
By the universal property of the quotient group, we know that $\overline{\phi}$ is both unique and well-defined. First suppose that $N=\ker\phi$. Let $gN\in\ker\overline{\phi}$. Then $\phi(g)=\overline{\phi}(gN)=e$, and hence $g\in\ker\phi=N$, implying that $gN=N$. Hence, $\ker\overline{\phi}=N$, implying that $\overline{\phi}$ is injective.

\noindent Now suppose that $\overline{\phi}$ is injective. That then means that $\ker\overline{\phi}=N$. Let $g\in\ker\phi$. Then $\overline{\phi}(gN)=\phi(g)=e$, implying that $gN=N$, or $g\in N$. Hence, $N=\ker\phi$.
\end{proof}

\begin{theorem}
Let $G$ be a group and let $N\trianglelefteq G$ be a normal subgroup. Let $q:G\to Q$ and $Q':G\to Q'$ be two quotient homomorphisms. Then there exists a unique isomorphism $h:Q\to Q'$ such that $q'=h\circ q$.
\end{theorem}
\begin{proof}
Let $\pi:G\to K$ be a quotient homomorphism. Let $\iota:N\hookrightarrow G$ be the inclusion homomorphism, and let $z:N\to G:n\mapsto e$ be the trivial homomorphism. Then $\pi\circ\iota=\pi\circ z\iff\pi(n)=e\forall n\in N\iff N\subseteq\ker\pi$, which is true by the definition of $\pi$. Let $\phi:G\to H$ be any homomorphism with $N\subseteq\ker\phi$, or in other words, with $\phi\circ\iota=\phi\circ z$. Then by the universal property, there exists a unique homomorphism $\overline{\phi}:K\to H$ such that $\phi=\overline{\phi}\circ\pi$.\[\begin{tikzcd}
N \arrow[r, shift left, "\iota"] \arrow[r, shift right, swap, "z"] & G \arrow[r, "\pi"] \arrow[dr, swap, "\forall \phi"] & K \arrow[d, "\exists! \overline{\phi}"] \\
& & H
\end{tikzcd}\]This is exactly equivalent to saying that $\pi$ is a coequalizer in the category of groups. Hence, the result follows by uniqueness.
\end{proof}
\subsection{Products}
\begin{definition}[Universal property of the direct product of groups]
Let $(G_\alpha)_{\alpha\in\mathcal{A}}$ be an indexed collection of groups. A group $G$ equipped with homomorphisms $\text{pr}_\alpha:G\to G_\alpha$ is a direct product of $(G_\alpha)_{\alpha\in\mathcal{A}}$ if for any group $H$ and homomorphisms $\phi_\alpha:H\to G_\alpha$, there exists a unique homomorphism $\Psi:H\to G$ such that $\text{pr}_\alpha\circ\Psi=\phi_\alpha\forall\alpha\in\mathcal{A}$.
\end{definition}
\begin{theorem}
Let $(G_\alpha)_{\alpha\in\mathcal{A}}$ be an indexed collection of groups. Let \[\prod_{\alpha\in\mathcal{A}}G_\alpha=\{g:\mathcal{A}\to\bigcup_{\alpha\in\mathcal  A}G_\alpha\mid \forall\alpha\in\mathcal{A}:g(\alpha)\in G_\alpha\}\] be a group equipped with homomorphisms $\text{pr}_\alpha:\prod_{\alpha\in\mathcal{A}}G_\alpha\to G_\alpha:g\mapsto g(\alpha)$, and with multiplication given by $(g\cdot h)(\alpha)=g(\alpha)h(\alpha)$. Then $\prod_{\alpha\in\mathcal{A}}G_\alpha$ is a direct product of $(G_\alpha)_{\alpha\in\mathcal{A}}$.
\end{theorem}
\begin{proof}
Let $H$ be another group and let $f_\beta:H\to G_\beta$ be a collection of homomorphisms. There is then a unique homomorphism $F:H\to\prod_{\alpha\in\mathcal{A}}G_\alpha$ with $\text{pr}_\beta\circ F=f_\beta$, namely the one given by $F(h)(\alpha)=f_\alpha(h)\forall h\in H,\alpha\in\mathcal{A}$.
\end{proof}
\begin{theorem}
Direct products of groups are unique up to unique isomorphism.
\end{theorem}
\begin{proof}
Direct products are products in the category of groups, and so uniqueness up to unique isomorphism follows.
\end{proof}
\subsection{Abelianization}
\begin{definition}[Universal property of the abelianization of a group]
Let $G$ be a group. An abelianization of $G$ is a pair $(G^\text{ab},\pi)$, where $G^\text{ab}$ is an abelian group and $\pi:G\to G^\text{ab}$ is a group homomorphism, such that given any abelian group $H$ and group homomorphism $\phi:G\to H$, there exists a unique group homomorphism $\phi^\text{ab}:G^\text{ab}\to H$ such that $\phi^\text{ab}\circ\pi=\phi$.
\end{definition}

\begin{theorem}
Let $G$ be a group, let $[G,G]$ be the commutator subgroup, and let $\pi:G\to G/[G,G]$ be the canonical quotient homomorphism. Then $(G/[G,G],\pi)$ is an abelianization of $G$.
\end{theorem}
\begin{proof}
We simply need to show that the commutator subgroup $[G,G]$ is contained within the kernel of $\phi$, since that will allow us to apply the universal property of the quotient group to obtain the unique group homomorphism $\phi^\text{ab}:G/[G,G]\to H$ such that $\phi^\text{ab}\circ\pi=\phi$. Indeed, let $x,y\in G$. Then $\phi([x,y])=\phi(x^{-1}y^{-1}xy)=\phi(x)^{-1}\phi(y)^{-1}\phi(x)\phi(y)=e$, so $[x,y]\in\ker\phi$. Hence, since $[G,G]$ is the subgroup generated by all commutators, it follows that $[G,G]\subseteq\ker\phi$, as required.
\end{proof}
\begin{theorem}
Let $G$ be a group, and let $(G^\text{ab},\pi)$ and $(G^{\text{ab}'},\pi')$ be two abelianizations of $G$. Then there exists a unique isomorphism $h:G^\text{ab}\to G^{\text{ab}'}$ such that $\pi'=h\circ\pi$.
\end{theorem}
\begin{proof}
By the universal property of the abelianization of a group, there exists a unique group homomorphism $h:G^\text{ab}\to G^{\text{ab}'}$ such that $\pi'=h\circ\pi$. Similarly, there exists a unique group homomorphism $g:G^{\text{ab}'}\to G^\text{ab}$ such that $\pi=g\circ\pi'$. Hence, $\pi'=(h\circ g)\circ\pi'$. However, we also have $\pi'=\text{id}_{G^{\text{ab}'}}\circ\pi'$, and hence, by uniqueness, we have $h\circ g=\text{id}_{G^{\text{ab}'}}$. Similarly, $g\circ h=\text{id}_{G^\text{ab}}$, implying that $h$ is an isomorphism.
\end{proof}

\begin{lemma}
Let $\phi:G\to H$ be a group homomorphism, and let $\pi_G:G\to G/[G,G]$, $\pi_H:H\to H/[H,H]$ be the quotient homomophisms of $G$ and $H$ into their respective abelianizations. Then there exists a group homomorphism $\phi^\text{ab}:G^\text{ab}\to H^\text{ab}$ given by $\phi^\text{ab}(g[G,G])=\phi(g)[H,H]\forall g[G,G]\in G^\text{ab}$, which is the unique group homomorphism $\psi:G^\text{ab}\to H^\text{ab}$ such that $\psi\circ\pi_G=\pi_H\circ\phi$.
\end{lemma}
\begin{proof}
Consider the map $\pi_H\circ\phi:G\to H^\text{ab}$. This is a map from a group to an abelian group, so by the universal property of the abelianization of a group, there exists a unique group homomorphism $\phi^\text{ab}:G^\text{ab}\to H^\text{ab}$ such that $\phi^\text{ab}\circ\pi_G=\pi_H\circ\phi$. That is, given any $g\in G$, we have \[\phi^\text{ab}(g[G,G])=\phi^\text{ab}\circ\pi_G(g)=\pi_H\circ\phi(g)=\phi(g)[H,H],\] as required.
\end{proof}

\begin{theorem}
There is a covariant functor, called the abelianization functor, from the category of groups to the category of abelian groups.
\end{theorem}
\begin{proof}
Let $F$ be the functor which maps groups to their abelianizations, and group homomorphisms to group homomorphisms between the abelianizations of the groups. That is, given groups $G$ and $H$ and a group homomorphism $\phi:G\to H$, we have $F(G)=G^\text{ab}$ and $F(\phi)=\phi^\text{ab}:G^\text{ab}\to H^\text{ab}:g[G,G]\mapsto\phi(g)[H,H]$.
\newline

\noindent We first show that $F(\text{id}_G)=\text{id}_{F(G)}$ for any group $G$. Indeed, $F(\text{id}_G)$ is given by $\text{id}^\text{ab}_{G^\text{ab}}:G^\text{ab}\to G^\text{ab}$, where $\text{id}^\text{ab}_{G^\text{ab}}(g[G,G])=\text{id}_G(g)[G,G]=g[G,G]\forall g\in G$. $F(\text{id}_G)$ is then the identity map on $G^\text{ab}=F(G)$, as required.
\newline

\noindent We now show that $F$ preserves composition. Let $G,H,K$ be groups and let $\phi:G\to H$ and $\psi:H\to K$ be group homomorphisms. Then \[F(\phi)=\phi^\text{ab}:G^\text{ab}\to H^\text{ab}:g[G,G]\mapsto \phi(g)[H,H],\] \[F(\psi)=\psi^\text{ab}:H^\text{ab}\to K^\text{ab}:h[H,H]\mapsto \psi(h)[K,K]\] and \[F(\psi\circ\phi)=(\psi\circ\phi)^\text{ab}:G^\text{ab}\to K^\text{ab}:g[G,G]\mapsto(\psi\circ\phi)(g)[K,K].\] Hence,\begin{align*}
    (\psi^\text{ab}\circ\phi^\text{ab})(g[G,G])&=\psi^\text{ab}(\phi(g)[H,H])\\&=\psi(\phi(g))[K,K]\\&=(\psi\circ\phi)(g)[K,K]\\&=(\psi\circ\phi)^\text{ab}(g[G,G]),
\end{align*} as required. Hence, $F$ is a covariant functor.
\end{proof}

\end{document}
