\documentclass{article}
\usepackage{graphicx} % Required for inserting images
\usepackage[english]{babel}
\usepackage{amssymb}
\usepackage{amsthm}
\usepackage{enumitem} 
\usepackage{amsmath}
\usepackage{amsfonts}
\usepackage{tikz-cd}
\usetikzlibrary{matrix}
\usepackage{mathtools}
\usepackage[a4paper, total={6in, 8in}]{geometry}
\usepackage{cite}
\graphicspath{ {./images/} }
\setcounter{section}{0}

\newtheorem{theorem}{Theorem}[section]
\newtheorem{definition}[theorem]{Definition}
\newtheorem{lemma}[theorem]{Lemma}
\newtheorem{proposition}[theorem]{Proposition}
\newtheorem{corollary}[theorem]{Corollary}
\newtheorem{example}[theorem]{Example}
\newtheorem{remark}[theorem]{Remark}
\newtheorem{exercise}{Exercise}[subsection]
\title{Algebra from a Categorical Perspective}
\author{Saxon Supple}
\date{February 2026}

\begin{document}

\maketitle
\section{Introduction}
This article will give categorical definitions of the standard objects from algebra, by way of defining universal properties which determine the objects up to unique isomorphism. For a given object $A$ of interest, the general procedure will be to first define an object $B$ as an object which satisfies a specific universal property, then show that $A$ satisfies that universal property, and then show that all pairs of objects $B$ which satisfy the universal property are related by a unique isomorphism. This will then demonstrate that the universal property determines $A$ up to unique isomorphism. We will then apply these universal properties to prove additional useful categorical results, such as the functoriality of the abelianization of a group.
\section{Sets}
\begin{definition}[Universal property of the quotient set]
Let $X$ be a set and let $\sim$ be an equivalence relation on $X$. A pair $(X',\pi:X\to X')$ is a quotient of $X$  by $\sim$ if given any $f:X\to Y$ such that $x_0\sim x_1\implies f(x_0)=f(x_1)\forall x_0,x_1\in X$, then there exists a unique $\overline{f}:X'\to Y$ such that $f=\overline{f}\circ \pi$.
\end{definition}
\begin{theorem}
The canonical projection $\pi:X\to X/\sim:x\mapsto[x]$ satisfies the universal property of the quotient set.
\end{theorem}
\begin{proof}
We define $\overline{f}:X/\sim\to Y$ by $\overline{f}(\pi(x))=f(x)$. This is well-defined, since if $x_0\sim x_1$, then $\overline{f}(\pi(x_0))=f(x_0)=f(x_1)=\overline{f}(\pi(x_1))$, and hence $\overline{f}$ is independent of representatives. Uniqueness is also evident from the definition of $\pi$.
\end{proof}
\begin{theorem}
Let $X$ be a set with an equivalence relation $\sim$. Let $(W,\pi_W)$ and $(Z,\pi_Z)$ be two pairs of sets, along with maps $\pi_W:X\to W$ and $\pi_Z:X\to Z$ such that both pairs satisfy the universal property of the quotient set. Then there exists a unique bijection $h:W\to Z$ such that $\pi_Z=h\circ\pi_W$.
\end{theorem}
\begin{proof}
Let $x_0\sim x_1$. Then $\pi_Z(x_0)=\pi_Z(x_1)$, by the definition of a quotient map. Hence, by the universal property of the quotient set, there exists a unique $h:W\to Z$ such that $\pi_Z=h\circ\pi_W$. Similarly, there exists a $g:Z\to W$ such that $\pi_W=g\circ\pi_Z$. Hence, $\pi_Z=(h\circ g)\circ\pi_Z$. However, we also have that $\pi_Z=\text{id}_Z\circ\pi_Z$, and hence, by uniqueness, $h\circ g=\text{id}_Z$. Similarly, $g\circ h=\text{id}_W$, implying that $h$ is a bijection, as required.
\end{proof}
\section{Groups}
\subsection{Quotient Groups}
\begin{definition}[Universal property of the quotient group]
Let $G$ be a group, and let $N\trianglelefteq G$ be a normal subgroup. A pair $(Q,q:G\to Q)$, consisting of a group $Q$ and a group homomorphism $q$, is a quotient of $G$ if $N\subseteq\ker q$ and given any group homomorphism $\phi:G\to H$ with $N\subseteq\ker \phi$, there exists a unique group homomorphism $\overline{\phi}:Q\to H$ such that $\phi=\overline{\phi}\circ q$.
\end{definition}
\begin{theorem} Let $G$ be a group and let $N\trianglelefteq G$ be a normal subgroup. Then the canonical quotient homomorphism $\pi:G\to G/N:g\mapsto gN$ is a quotient map.
\end{theorem}
\begin{proof}
Let $\sim$ be an equivalence relation on $G$ which identifies the fibres of $\pi$; that is, $x\sim y\iff xN=yN$. If $\pi(x)=\pi(y)$, then $x$ and $y$ are in the same coset, so there exists some $n\in N$ such that $x=yn$. Hence, since $N\subseteq\ker\phi$, we have $\phi(x)=\phi(yn)=\phi(y)\phi(n)=\phi(y)$. Then by the universal property of the quotient set, there exists a unique map $\overline{\phi}:G/\sim\to H$ such that $\phi=\overline{\phi}\circ\pi$. In particular, $\overline{\phi}$ is defined as $\overline{\phi}([x])=\overline{\phi}(xN)=\phi(x)\forall x$, and hence $\overline{\phi}$ is simply a map on the cosets of $N$. Furthermore, given $x,y\in G$, we have $\overline{\phi}(xyN)=\phi(xy)=\phi(x)\phi(y)=\overline{\phi}(xN)\overline{\phi}(yN)$, and hence $\overline{\phi}$ is a group homomorphism.
\end{proof}

\begin{corollary}
Let $\phi:G\to H$ be a group homomorphism, and let $N\trianglelefteq G$ be a normal subgroup such that $N\subseteq\ker\phi$. Then $N=\ker\phi$ if and only if $\overline{\phi}:G/N\to H$ is injective.
\end{corollary}
\begin{proof}
By the universal property of the quotient group, we know that $\overline{\phi}$ is both unique and well-defined. First suppose that $N=\ker\phi$. Let $gN\in\ker\overline{\phi}$. Then $\phi(g)=\overline{\phi}(gN)=e$, and hence $g\in\ker\phi=N$, implying that $gN=N$. Hence, $\ker\overline{\phi}=N$, implying that $\overline{\phi}$ is injective.

\noindent Now suppose that $\overline{\phi}$ is injective. That then means that $\ker\overline{\phi}=N$. Let $g\in\ker\phi$. Then $\overline{\phi}(gN)=\phi(g)=e$, implying that $gN=N$, or $g\in N$. Hence, $N=\ker\phi$.
\end{proof}

\begin{theorem}
Let $G$ be a group and let $N\trianglelefteq G$ be a normal subgroup. Let $q:G\to Q$ and $Q':G\to Q'$ be two quotient homomorphisms. Then there exists a unique isomorphism $h:Q\to Q'$ such that $q'=h\circ q$.
\end{theorem}
\begin{proof}
By the universal property of the quotient group, there exists a unique group homomorphism $h:Q\to Q'$ such that $q'=h\circ q$, and similarly, there exists a unique group homomorphism $g:Q'\to Q$ such that $q=g\circ q'$. Hence, $q'=(h\circ g)\circ q'$. However, we also have $q'=\text{id}_{Q'}\circ q'$, and so the universal property implies that $h\circ g=\text{id}_{Q'}$. Similarly, $g\circ h=\text{id}_Q$, and hence $h$ is an isomorphism.
\end{proof}

\subsection{Abelianization}
\begin{definition}[Universal property of the abelianization of a group]
Let $G$ be a group. An abelianization of $G$ is a pair $(G^\text{ab},\pi)$, where $G^\text{ab}$ is an abelian group and $\pi:G\to G^\text{ab}$ is a group homomorphism, such that given any abelian group $H$ and group homomorphism $\phi:G\to H$, there exists a unique group homomorphism $\phi^\text{ab}:G^\text{ab}\to H$ such that $\phi^\text{ab}\circ\pi=\phi$.
\end{definition}

\begin{theorem}
Let $G$ be a group, let $[G,G]$ be the commutator subgroup, and let $\pi:G\to G/[G,G]$ be the canonical quotient homomorphism. Then $(G/[G,G],\pi)$ is an abelianization of $G$.
\end{theorem}
\begin{proof}
We simply need to show that the commutator subgroup $[G,G]$ is contained within the kernel of $\phi$, since that will allow us to apply the universal property of the quotient group to obtain the unique group homomorphism $\phi^\text{ab}:G/[G,G]\to H$ such that $\phi^\text{ab}\circ\pi=\phi$. Indeed, let $x,y\in G$. Then $\phi([x,y])=\phi(x^{-1}y^{-1}xy)=\phi(x)^{-1}\phi(y)^{-1}\phi(x)\phi(y)=e$, so $[x,y]\in\ker\phi$. Hence, since $[G,G]$ is the subgroup generated by all commutators, it follows that $[G,G]\subseteq\ker\phi$, as required.
\end{proof}
\begin{theorem}
Let $G$ be a group, and let $(G^\text{ab},\pi)$ and $(G^{\text{ab}'},\pi')$ be two abelianizations of $G$. Then there exists a unique isomorphism $h:G^\text{ab}\to G^{\text{ab}'}$ such that $\pi'=h\circ\pi$.
\end{theorem}
\begin{proof}
By the universal property of the abelianization of a group, there exists a unique group isomorphism $h:G^\text{ab}\to G^{\text{ab}'}$ such that $\pi'=h\circ\pi$. Similarly, there exists a unique group isomorphism $g:G^{\text{ab}'}\to G^\text{ab}$ such that $\pi=g\circ\pi'$. Hence, $\pi'=(h\circ g)\circ\pi'$. However, we also have $\pi'=\text{id}_{G^{\text{ab}'}}\circ\pi'$, and hence, by uniqueness, we have $h\circ g=\text{id}_{G^{\text{ab}'}}$. Similarly, $g\circ h=\text{id}_{G^\text{ab}}$, implying that $h$ is an isomorphism.
\end{proof}

\begin{lemma}
Let $\phi:G\to H$ be a group homomorphism, and let $\pi_G:G\to G/[G,G]$, $\pi_H:H\to H/[H,H]$ be the quotient homomophisms of $G$ and $H$ into their respective abelianizations. Then there exists a group homomorphism $\phi^\text{ab}:G^\text{ab}\to H^\text{ab}$ given by $\phi^\text{ab}(g[G,G])=\phi(g)[H,H]\forall g[G,G]\in G^\text{ab}$, which is the unique group homomorphism $\psi:G^\text{ab}\to H^\text{ab}$ such that $\psi\circ\pi_G=\pi_H\circ\phi$.
\end{lemma}
\begin{proof}
Consider the map $\pi_H\circ\phi:G\to H^\text{ab}$. This is a map from a group to an abelian group, so by the universal property of the abelianization of a group, there exists a unique group homomorphism $\phi^\text{ab}:G^\text{ab}\to H^\text{ab}$ such that $\phi^\text{ab}\circ\pi_G=\pi_H\circ\phi$. That is, given any $g\in G$, we have \[\phi^\text{ab}(g[G,G])=\phi^\text{ab}\circ\pi_G(g)=\pi_H\circ\phi(g)=\phi(g)[H,H],\] as required.
\end{proof}

\begin{theorem}
There is a covariant functor, called the abelianization functor, from the category of groups to the category of abelian groups.
\end{theorem}
\begin{proof}
Let $F$ be the functor which maps groups to their abelianizations, and group homomorphisms to group homomorphisms between the abelianizations of the groups. That is, given groups $G$ and $H$ and a group homomorphism $\phi:G\to H$, we have $F(G)=G^\text{ab}$ and $F(\phi)=\phi^\text{ab}:G^\text{ab}\to H^\text{ab}:g[G,G]\mapsto\phi(g)[H,H]$.
\newline

\noindent We first show that $F(\text{id}_G)=\text{id}_{F(G)}$ for any group $G$. Indeed, $F(\text{id}_G)$ is given by $\text{id}^\text{ab}_{G^\text{ab}}:G^\text{ab}\to G^\text{ab}$, where $\text{id}^\text{ab}_{G^\text{ab}}(g[G,G])=\text{id}_G(g)[G,G]=g[G,G]\forall g\in G$. $F(\text{id}_G)$ is then the identity map on $G^\text{ab}=F(G)$, as required.
\newline

\noindent We now show that $F$ preserves composition. Let $G,H,K$ be groups and let $\phi:G\to H$ and $\psi:H\to K$ be group homomorphisms. Then \[F(\phi)=\phi^\text{ab}:G^\text{ab}\to H^\text{ab}:g[G,G]\mapsto \phi(g)[H,H],\] \[F(\psi)=\psi^\text{ab}:H^\text{ab}\to K^\text{ab}:h[H,H]\mapsto \psi(h)[K,K]\] and \[F(\psi\circ\phi)=(\psi\circ\phi)^\text{ab}:G^\text{ab}\to K^\text{ab}:g[G,G]\mapsto(\psi\circ\phi)(g)[K,K].\] Hence,\begin{align*}
    (\psi^\text{ab}\circ\phi^\text{ab})(g[G,G])&=\psi^\text{ab}(\phi(g)[H,H])\\&=\psi(\phi(g))[K,K]\\&=(\psi\circ\phi)(g)[K,K]\\&=(\psi\circ\phi)^\text{ab}(g[G,G]),
\end{align*} as required. Hence, $F$ is a covariant functor.
\end{proof}

\end{document}
